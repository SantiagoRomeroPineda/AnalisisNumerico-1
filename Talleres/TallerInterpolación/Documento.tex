
\documentclass[12pt]{article}
\usepackage{amsmath}
\usepackage{latexsym}
\usepackage{amsfonts}
\usepackage[normalem]{ulem}
\usepackage{soul}
\usepackage{array}
\usepackage{amssymb}
\usepackage{extarrows}
\usepackage{graphicx}
\usepackage[backend=biber,
style=numeric,
sorting=none,
isbn=false,
doi=false,
url=false,
]{biblatex}\addbibresource{bibliography.bib}

\usepackage{subfig}
\usepackage{wrapfig}
\usepackage{wasysym}
\usepackage{enumitem}
\usepackage{adjustbox}
\usepackage{ragged2e}
\usepackage[svgnames,table]{xcolor}
\usepackage{tikz}
\usepackage{longtable}
\usepackage{changepage}
\usepackage{setspace}
\usepackage{hhline}
\usepackage{multicol}
\usepackage{tabto}
\usepackage{float}
\usepackage{multirow}
\usepackage{makecell}
\usepackage{fancyhdr}
\usepackage[toc,page]{appendix}
\usepackage[hidelinks]{hyperref}
\usetikzlibrary{shapes.symbols,shapes.geometric,shadows,arrows.meta}
\tikzset{>={Latex[width=1.5mm,length=2mm]}}
\usepackage{flowchart}\usepackage[paperheight=11.0in,paperwidth=8.5in,left=1.18in,right=1.18in,top=0.98in,bottom=0.98in,headheight=1in]{geometry}
\usepackage[utf8]{inputenc}
\usepackage[T1]{fontenc}
\TabPositions{0.49in,0.98in,1.47in,1.96in,2.45in,2.94in,3.43in,3.92in,4.41in,4.9in,5.39in,5.88in,}

\urlstyle{same}

\renewcommand{\_}{\kern-1.5pt\textunderscore\kern-1.5pt}

 %%%%%%%%%%%%  Set Depths for Sections  %%%%%%%%%%%%%%

% 1) Section
% 1.1) SubSection
% 1.1.1) SubSubSection
% 1.1.1.1) Paragraph
% 1.1.1.1.1) Subparagraph


\setcounter{tocdepth}{5}
\setcounter{secnumdepth}{5}


 %%%%%%%%%%%%  Set Depths for Nested Lists created by \begin{enumerate}  %%%%%%%%%%%%%%


\setlistdepth{9}
\renewlist{enumerate}{enumerate}{9}
		\setlist[enumerate,1]{label=\arabic*)}
		\setlist[enumerate,2]{label=\alph*)}
		\setlist[enumerate,3]{label=(\roman*)}
		\setlist[enumerate,4]{label=(\arabic*)}
		\setlist[enumerate,5]{label=(\Alph*)}
		\setlist[enumerate,6]{label=(\Roman*)}
		\setlist[enumerate,7]{label=\arabic*}
		\setlist[enumerate,8]{label=\alph*}
		\setlist[enumerate,9]{label=\roman*}

\renewlist{itemize}{itemize}{9}
		\setlist[itemize]{label=$\cdot$}
		\setlist[itemize,1]{label=\textbullet}
		\setlist[itemize,2]{label=$\circ$}
		\setlist[itemize,3]{label=$\ast$}
		\setlist[itemize,4]{label=$\dagger$}
		\setlist[itemize,5]{label=$\triangleright$}
		\setlist[itemize,6]{label=$\bigstar$}
		\setlist[itemize,7]{label=$\blacklozenge$}
		\setlist[itemize,8]{label=$\prime$}

\setlength{\topsep}{0pt}\setlength{\parskip}{8.04pt}
\setlength{\parindent}{0pt}

 %%%%%%%%%%%%  This sets linespacing (verticle gap between Lines) Default=1 %%%%%%%%%%%%%%


\renewcommand{\arraystretch}{1.3}


%%%%%%%%%%%%%%%%%%%% Document code starts here %%%%%%%%%%%%%%%%%%%%



\begin{document}
\begin{Center}
{\fontsize{14pt}{16.8pt}\selectfont \textbf{Taller interpolación}\par}
\end{Center}\par

Angela Sofia Moreno\tab \tab \tab \tab \tab \ \ \  Julio Andrés Mejía\par

\tab {\fontsize{9pt}{10.8pt}\selectfont \textcolor[HTML]{484644}{moreno\_angela@javeriana.edu.co\par}}\ \ \ \ \ \ \ \ \ \ \ \ \ \ \ \ \ \ \ \ \ \ \ \ \ \ \ \ \ \ \ \ \ \ \ \ \ \ \ \ \ \ \ \  {\fontsize{9pt}{10.8pt}\selectfont \textcolor[HTML]{484644}{julio.mejia@javeriana.edu.co}\par}\par

Anggie\ Carolina\ Correa   \tab \tab \tab \tab Brayan Estiben Giraldo\par

{\fontsize{9pt}{10.8pt}\selectfont \textcolor[HTML]{484644}{\ \ \ \ \ \ \ \ \ \ \ \  anggie.correa@javeriana.edu.co\par}}\tab \ \ \ \ \ \ \ \ \ \ \ \ \ \ \ \ \ \ \ \ \ \ \ \ \ \ \ \ \ \ \ \ \ \ \ \ \  {\fontsize{9pt}{10.8pt}\selectfont \textcolor[HTML]{484644}{bestiben.giraldol@javeriana.edu.co}\par}\par

\begin{Center}
\textit{Abril 2020}
\end{Center}\par


\vspace{\baselineskip}
\begin{enumerate}[label*={\fontsize{14pt}{14pt}\selectfont \textbf{\arabic*.}}]
	\item {\fontsize{14pt}{16.8pt}\selectfont \textbf{Primer punto}\par}\par

\textbf{1.1. Problema}\par

Dados los n + 1 puntos distintos (xi,yi). Demuestre que el polinomio interpolante que incluye a todos los puntos es único\par

\textbf{1.2. Solución}\par

Sean x1, x2, . . ., xn algunos números diferentes por pares y sean y1, y2. . ., yn algunos números. Entonces existe un único polinomio P de grado (Grado n1) tal que: \par

P(xj) = yj (j=1, . . . , n).\par

Las incógnitas del problema son los coeficientesc0, . . . , cn1 del polinomio\par

P: (x) =c0+c1x+. . .+cn1xn1=n1j=0cjxj.\par


\vspace{\baselineskip}

\vspace{\baselineskip}
	\item {\fontsize{14pt}{16.8pt}\selectfont \textbf{Segundo punto}\par}\par

\textbf{2.1. Problema}\par

Construya un polinomio de grado tres que pase por: (0,10),(1,15),(2,5) y que la tangente sea igual a 1 en x0\par

\textbf{2.2. Solución}\par

Entradas:\ \  x : Puntos en X. y : Puntos en Y.\par

Salidas:\  f : Polinomio resultado de interpolar.\par

Para esta solución se utilizó la librería Scipy de Python y la función:\par

f $ \leftarrow $  interpolate.CubicSpline(x,y)\par


\vspace{\baselineskip}


%%%%%%%%%%%%%%%%%%%% Figure/Image No: 1 starts here %%%%%%%%%%%%%%%%%%%%

\begin{figure}[H]
	\begin{Center}
		\includegraphics[width=3.97in,height=3.4in]{./media/image1.png}
	\end{Center}
\end{figure}


%%%%%%%%%%%%%%%%%%%% Figure/Image No: 1 Ends here %%%%%%%%%%%%%%%%%%%%

\par


\vspace{\baselineskip}
	\item {\fontsize{14pt}{16.8pt}\selectfont \textbf{Tercer punto}\par}\par

\textbf{3.1. Problema}\par

Construya un polinomio del menor grado que interpole una función f(x) en los siguientes datos: \par

f(1)\ =\ 2;\ \  f(2) = 6;   f’(1)\ =\ 3;   f’(2)\ =\ 7;   f’’(2) = 8\par


\vspace{\baselineskip}
	\item {\fontsize{14pt}{16.8pt}\selectfont \textbf{Cuarto punto}\par}\par

\textbf{4.1. Problema}\par

Con la función f(x) = lnx construya la interpolación de diferencias divididas en\par

\ x0\ = 1;   x1 = 2 y estime el error en [1,2]\par

\begin{enumerate}[label*={\fontsize{12pt}{12pt}\selectfont \textbf{\arabic*.}}]
	\item \textbf{Solución}\par



%%%%%%%%%%%%%%%%%%%% Figure/Image No: 2 starts here %%%%%%%%%%%%%%%%%%%%

\begin{figure}[H]
	\begin{Center}
		\includegraphics[width=2.21in,height=0.48in]{./media/image2.png}
	\end{Center}
\end{figure}


%%%%%%%%%%%%%%%%%%%% Figure/Image No: 2 Ends here %%%%%%%%%%%%%%%%%%%%

\par



%%%%%%%%%%%%%%%%%%%% Figure/Image No: 3 starts here %%%%%%%%%%%%%%%%%%%%

\begin{figure}[H]
	\begin{Center}
		\includegraphics[width=3.38in,height=1.69in]{./media/image3.png}
	\end{Center}
\end{figure}


%%%%%%%%%%%%%%%%%%%% Figure/Image No: 3 Ends here %%%%%%%%%%%%%%%%%%%%

\par


\vspace{\baselineskip}

\end{enumerate}
	\item {\fontsize{14pt}{16.8pt}\selectfont \textbf{Quinto punto}\par}\par

\textbf{5.1. Problema}\par

Utilice la interpolación de splines cúbicos para el problema de la mano \par

\begin{enumerate}[label*={\fontsize{12pt}{12pt}\selectfont \textbf{\arabic*.}}]
	\item \textbf{Solución}\par

Entradas: x : Puntos en X. y : Puntos en Y.\par

Salidas: lag\_pol : Polinomio interpolador\par


\vspace{\baselineskip}
Realizamos el modelado de una de las manos de los integrantes del grupo\par


\vspace{\baselineskip}


%%%%%%%%%%%%%%%%%%%% Figure/Image No: 4 starts here %%%%%%%%%%%%%%%%%%%%

\begin{figure}[H]
	\begin{Center}
		\includegraphics[width=4.61in,height=4.0in]{./media/image4.png}
	\end{Center}
\end{figure}


%%%%%%%%%%%%%%%%%%%% Figure/Image No: 4 Ends here %%%%%%%%%%%%%%%%%%%%

\par


\vspace{\baselineskip}

\vspace{\baselineskip}

\end{enumerate}
	\item {\fontsize{14pt}{16.8pt}\selectfont \textbf{Sexto punto}\par}\par

\textbf{6.1 Problema}\par

Sea f(x) = tan(x) utilice la partición de la forma xi = $ \delta $ k para implementar una interpolación para n=10 puntos y encuentre el valor $ \delta $  que minimice el error\par


\vspace{\baselineskip}
\begin{enumerate}[label*={\fontsize{12pt}{12pt}\selectfont \arabic*.}]
	\item \textbf{Solución}
\end{enumerate}\par

Primero se debe definir el inicio, luego los pasos o intervalos y luego generar los 10 puntos\par

Luego se calcula el sigma que minimiza el error\par

\begin{adjustwidth}{0.21in}{0.0in}
El sigma que minimiza el error es:\  0.31999999999999984\par

\end{adjustwidth}

\begin{adjustwidth}{0.21in}{0.0in}
La función luce así:\par

\end{adjustwidth}



%%%%%%%%%%%%%%%%%%%% Figure/Image No: 5 starts here %%%%%%%%%%%%%%%%%%%%

\begin{figure}[H]
	\begin{Center}
		\includegraphics[width=4.39in,height=3.7in]{./media/image5.png}
	\end{Center}
\end{figure}


%%%%%%%%%%%%%%%%%%%% Figure/Image No: 5 Ends here %%%%%%%%%%%%%%%%%%%%

\par


\vspace{\baselineskip}
	\item {\fontsize{14pt}{16.8pt}\selectfont \textbf{Séptimo punto}\par}\par

\textbf{7.1. Problema}\par

Sea f(x) =  \( e^{x} \)  en el intervalo de [0,1] utilice el método de lagrange y determine el tamaño del paso que me produzca un error por debajo de 10$-$ 5. \par


\vspace{\baselineskip}
¿Es posible utilizar el polinomio de Taylor para interpolar en este caso? Verifique su respuesta.\par


\vspace{\baselineskip}
\begin{enumerate}[label*={\fontsize{12pt}{12pt}\selectfont \textbf{\arabic*.}}]
	\item \textbf{Solución}\par

Utilizando lagrange se obtienen los pasos que me producen un error menor de 10-5 n un intervalo [0,1]\par


\vspace{\baselineskip}
x $ \leftarrow $  numpy.arange(0,1,ini) 9\par

y $ \leftarrow $  ex 10 \par

f $ \leftarrow $  lagrange(x,y)\par


\vspace{\baselineskip}
El\ paso que minimiza el error es :  0.49999999999999956\par


\vspace{\baselineskip}
Si es posible utilizar Taylor para esto, pero podemos decir que el polinomio no es un buen interpolador, esto debido a que no se logra representar el comportamiento de la función original por medio del polinomio. Si bien se logran ver algunas regiones (puntos) donde se llega a ser el polinomio bastante cercano a la función original, en otras regiones se aleja bastante los valores originales, esto también gracias a que el polinomio es de grado 4.\par

A continuación, podemos observar la evaluación del polinomio con su error\par



%%%%%%%%%%%%%%%%%%%% Figure/Image No: 6 starts here %%%%%%%%%%%%%%%%%%%%

\begin{figure}[H]
	\begin{Center}
		\includegraphics[width=3.91in,height=1.98in]{./media/image6.png}
	\end{Center}
\end{figure}


%%%%%%%%%%%%%%%%%%%% Figure/Image No: 6 Ends here %%%%%%%%%%%%%%%%%%%%

\par


\end{enumerate}
	\item {\fontsize{14pt}{16.8pt}\selectfont \textbf{Octavo punto}\par}
\end{enumerate}\par

\textbf{8.1. Problema}\par

Considere el comportamiento de gases no ideales se describe a menudo con la ecuación viral de estado. los siguientes datos para el nitrógeno N2\par



%%%%%%%%%%%%%%%%%%%% Figure/Image No: 7 starts here %%%%%%%%%%%%%%%%%%%%

\begin{figure}[H]
	\begin{Center}
		\includegraphics[width=4.24in,height=0.51in]{./media/image7.png}
	\end{Center}
\end{figure}


%%%%%%%%%%%%%%%%%%%% Figure/Image No: 7 Ends here %%%%%%%%%%%%%%%%%%%%

\par

Donde T es la temperatura [K] y B es el segundo coeficiente viral. El comportamiento de gases no ideales se describe a menudo con la ecuación viral de estado\par



%%%%%%%%%%%%%%%%%%%% Figure/Image No: 8 starts here %%%%%%%%%%%%%%%%%%%%

\begin{figure}[H]
	\begin{Center}
		\includegraphics[width=2.15in,height=0.56in]{./media/image8.png}
	\end{Center}
\end{figure}


%%%%%%%%%%%%%%%%%%%% Figure/Image No: 8 Ends here %%%%%%%%%%%%%%%%%%%%

\par

Donde P es la presión, V el volumen molar del gas, T es la temperatura Kelvin y R es la constante de gas ideal. Los coeficientes B = B(T), C = C(T), son el segundo y tercer coeficiente viral, respectivamente. En la práctica se usa la serie truncada para aproximar\par



%%%%%%%%%%%%%%%%%%%% Figure/Image No: 9 starts here %%%%%%%%%%%%%%%%%%%%

\begin{figure}[H]
	\begin{Center}
		\includegraphics[width=1.24in,height=0.52in]{./media/image9.png}
	\end{Center}
\end{figure}


%%%%%%%%%%%%%%%%%%%% Figure/Image No: 9 Ends here %%%%%%%%%%%%%%%%%%%%

\par

En la siguiente figura se muestra cómo se distribuye la variable B a lo largo de la temperatura.\par

a) Determine un polinomio interpolante para este caso \par

b) Utilizando el resultado anterior calcule el segundo y tercer coeficiente viral a 450K. \par

c) Grafique los puntos y el polinomio que ajusta \par

d) Utilice la interpolación de Lagrange y escriba el polinomio interpolante \par

e) Compare su resultado con la serie truncada (modelo teórico), ¿cuál aproximación es mejor por qué?\par


\vspace{\baselineskip}
\textbf{8.2. Solución}\par

 8.2.1 Punto a.\par

 Polinomio obtenido al utilizar la función $``$poly\_calc$"$  de la Librería $"$ PolynomF$"$ . No se  utilizó el primer punto (100,-160) ya que se consideró un punto atípico, causando que no se \  genere el polinomio de manera correcta\par

{\fontsize{11pt}{13.2pt}\selectfont \ \ \ \ \ \ \ \ \ \ \ \ \ \ \ \  y = -1061.1 + 11.67475$\ast$ x - 0.04678125$\ast$ x$ \string^ $ 2 + 8.0175e-05$\ast$ x$ \string^ $ 3 - 4.9375e-08$\ast$ x$ \string^ $ 4\par}\par


\vspace{\baselineskip}

\vspace{\baselineskip}

\vspace{\baselineskip}
{\fontsize{11pt}{13.2pt}\selectfont \ \ \ \ \  \ \ \ \ \ \ \ \   8.2.2 Punto b.\par}\par


\vspace{\baselineskip}
\tab {\fontsize{11pt}{13.2pt}\selectfont Al evaluar el polinomio con x = 450 dio como resultado 0.59765625\par}\par


\vspace{\baselineskip}
{\fontsize{11pt}{13.2pt}\selectfont \ \ \ \ \  \ \ \ \ \ \ \ \ \  8.2.3 Punto c.\par}\par


\vspace{\baselineskip}
{\fontsize{11pt}{13.2pt}\selectfont \tab Se graficó tanto datos reales (cuadrados rojos) como los datos \par}\par

{\fontsize{11pt}{13.2pt}\selectfont \tab interpolados (cuadrados azules)\par}\par

\tab 
\vspace{\baselineskip}

%%%%%%%%%%%%%%%%%%%% Figure/Image No: 10 starts here %%%%%%%%%%%%%%%%%%%%

\begin{figure}[H]
	\begin{Center}
		\includegraphics[width=4.24in,height=2.38in]{./media/image10.png}
	\end{Center}
\end{figure}


%%%%%%%%%%%%%%%%%%%% Figure/Image No: 10 Ends here %%%%%%%%%%%%%%%%%%%%

\par



%%%%%%%%%%%%%%%%%%%% Figure/Image No: 11 starts here %%%%%%%%%%%%%%%%%%%%

\begin{figure}[H]
	\begin{Center}
		\includegraphics[width=4.07in,height=2.27in]{./media/image11.png}
	\end{Center}
\end{figure}


%%%%%%%%%%%%%%%%%%%% Figure/Image No: 11 Ends here %%%%%%%%%%%%%%%%%%%%

\par

\ \ \ \ \ \ \ \ \ \  \ \ \ \ \   8.2.4 Punto d.\par

\tab \  Utilizando la interpolación de lagrange se obtiene la siguiente tabla de datos:\par



%%%%%%%%%%%%%%%%%%%% Figure/Image No: 12 starts here %%%%%%%%%%%%%%%%%%%%

\begin{figure}[H]
	\begin{Center}
		\includegraphics[width=3.42in,height=1.77in]{./media/image12.png}
	\end{Center}
\end{figure}


%%%%%%%%%%%%%%%%%%%% Figure/Image No: 12 Ends here %%%%%%%%%%%%%%%%%%%%

\par



%%%%%%%%%%%%%%%%%%%% Figure/Image No: 13 starts here %%%%%%%%%%%%%%%%%%%%

\begin{figure}[H]
	\begin{Center}
		\includegraphics[width=3.21in,height=0.41in]{./media/image13.png}
	\end{Center}
\end{figure}


%%%%%%%%%%%%%%%%%%%% Figure/Image No: 13 Ends here %%%%%%%%%%%%%%%%%%%%

\par


\vspace{\baselineskip}

\vspace{\baselineskip}
\begin{adjustwidth}{0.49in}{0.0in}
Seguido de tener los L, se procede a multiplicar con su respectivo y. Para terminar, se suma todas las L, dando como resultado el polinomio:\par

\end{adjustwidth}


\vspace{\baselineskip}
\begin{adjustwidth}{0.49in}{0.0in}
y = $-$ 4,3750e$-$ 09 $\ast$ x$ \string^ $ 4 + 8,175e$-$ 06$\ast$  x$ \string^ $ 3 $-$  0,00583$\ast$  x$ \string^ $ 2 + 1,95475x $-$  251,1\par

\end{adjustwidth}


\vspace{\baselineskip}
\begin{adjustwidth}{0.49in}{0.0in}
\ \  8.2.5 Punto e\par

\end{adjustwidth}

\begin{adjustwidth}{0.49in}{0.0in}
\tab Al realizar la interpolación por diferentes métodos se puede mirar los cambios que tienen estos métodos y cuál de ellos es más cercano a los datos reales.\par

\end{adjustwidth}


\vspace{\baselineskip}


%%%%%%%%%%%%%%%%%%%% Figure/Image No: 14 starts here %%%%%%%%%%%%%%%%%%%%

\begin{figure}[H]
	\begin{Center}
		\includegraphics[width=3.29in,height=1.36in]{./media/image14.png}
	\end{Center}
\end{figure}


%%%%%%%%%%%%%%%%%%%% Figure/Image No: 14 Ends here %%%%%%%%%%%%%%%%%%%%

\par

\begin{adjustwidth}{0.49in}{0.0in}
Al ver la tabla comparativa de los datos obtenidos por las interpolaciones anteriores, podemos notar como la interpolación común representa correctamente el comportamiento de los datos reales. Por otro lado, por medio de la interpolación de lagrange se puede notar que si bien llega a tener unos puntos muy cercanos a los reales (problemas dados por errores de redondeo de valores), existen otros puntos los cuales se alejan bastante de los datos verdaderos. Se puede apreciar mejor esto al graficar todos los métodos.\par

\end{adjustwidth}

\begin{adjustwidth}{0.49in}{0.0in}
Con esto podemos decir que el mejor método de interpolación es por el método común de interpolación\par

\end{adjustwidth}


\vspace{\baselineskip}


%%%%%%%%%%%%%%%%%%%% Figure/Image No: 15 starts here %%%%%%%%%%%%%%%%%%%%

\begin{figure}[H]
	\begin{Center}
		\includegraphics[width=4.18in,height=2.47in]{./media/image15.png}
	\end{Center}
\end{figure}


%%%%%%%%%%%%%%%%%%%% Figure/Image No: 15 Ends here %%%%%%%%%%%%%%%%%%%%

\par


\printbibliography
\end{document}