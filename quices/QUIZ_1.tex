
\documentclass[12pt]{article}
\usepackage{amsmath}
\usepackage{latexsym}
\usepackage{amsfonts}
\usepackage[normalem]{ulem}
\usepackage{soul}
\usepackage{array}
\usepackage{amssymb}
\usepackage{extarrows}
\usepackage{graphicx}
\usepackage[backend=biber,
style=numeric,
sorting=none,
isbn=false,
doi=false,
url=false,
]{biblatex}\addbibresource{bibliography.bib}

\usepackage{subfig}
\usepackage{wrapfig}
\usepackage{wasysym}
\usepackage{enumitem}
\usepackage{adjustbox}
\usepackage{ragged2e}
\usepackage[svgnames,table]{xcolor}
\usepackage{tikz}
\usepackage{longtable}
\usepackage{changepage}
\usepackage{setspace}
\usepackage{hhline}
\usepackage{multicol}
\usepackage{tabto}
\usepackage{float}
\usepackage{multirow}
\usepackage{makecell}
\usepackage{fancyhdr}
\usepackage[toc,page]{appendix}
\usepackage[hidelinks]{hyperref}
\usetikzlibrary{shapes.symbols,shapes.geometric,shadows,arrows.meta}
\tikzset{>={Latex[width=1.5mm,length=2mm]}}
\usepackage{flowchart}\usepackage[paperheight=11.0in,paperwidth=8.5in,left=1.18in,right=1.18in,top=0.98in,bottom=0.98in,headheight=1in]{geometry}
\usepackage[utf8]{inputenc}
\usepackage[T1]{fontenc}
\TabPositions{0.49in,0.98in,1.47in,1.96in,2.45in,2.94in,3.43in,3.92in,4.41in,4.9in,5.39in,5.88in,}

\urlstyle{same}

\renewcommand{\_}{\kern-1.5pt\textunderscore\kern-1.5pt}

 %%%%%%%%%%%%  Set Depths for Sections  %%%%%%%%%%%%%%

% 1) Section
% 1.1) SubSection
% 1.1.1) SubSubSection
% 1.1.1.1) Paragraph
% 1.1.1.1.1) Subparagraph


\setcounter{tocdepth}{5}
\setcounter{secnumdepth}{5}


 %%%%%%%%%%%%  Set Depths for Nested Lists created by \begin{enumerate}  %%%%%%%%%%%%%%


\setlistdepth{9}
\renewlist{enumerate}{enumerate}{9}
		\setlist[enumerate,1]{label=\arabic*)}
		\setlist[enumerate,2]{label=\alph*)}
		\setlist[enumerate,3]{label=(\roman*)}
		\setlist[enumerate,4]{label=(\arabic*)}
		\setlist[enumerate,5]{label=(\Alph*)}
		\setlist[enumerate,6]{label=(\Roman*)}
		\setlist[enumerate,7]{label=\arabic*}
		\setlist[enumerate,8]{label=\alph*}
		\setlist[enumerate,9]{label=\roman*}

\renewlist{itemize}{itemize}{9}
		\setlist[itemize]{label=$\cdot$}
		\setlist[itemize,1]{label=\textbullet}
		\setlist[itemize,2]{label=$\circ$}
		\setlist[itemize,3]{label=$\ast$}
		\setlist[itemize,4]{label=$\dagger$}
		\setlist[itemize,5]{label=$\triangleright$}
		\setlist[itemize,6]{label=$\bigstar$}
		\setlist[itemize,7]{label=$\blacklozenge$}
		\setlist[itemize,8]{label=$\prime$}

\setlength{\topsep}{0pt}\setlength{\parskip}{8.04pt}
\setlength{\parindent}{0pt}

 %%%%%%%%%%%%  This sets linespacing (verticle gap between Lines) Default=1 %%%%%%%%%%%%%%


\renewcommand{\arraystretch}{1.3}


%%%%%%%%%%%%%%%%%%%% Document code starts here %%%%%%%%%%%%%%%%%%%%



\begin{document}
\begin{Center}
{\fontsize{14pt}{16.8pt}\selectfont \textbf{QUIZ 1}\par}
\end{Center}\par

\textit{Angela Sofia Moreno Rodriguez}\par

\textbf{Punto 1:}\par

Dados los n + 1 puntos distintos (xi , yi) el polinomio interpolante que incluye a todos los puntos es único\par

\textbf{Solución:}\par

Sean x1, x2, . . ., xn algunos números diferentes por pares y sean y1, y2. . ., yn algunos números. Entonces existe un único polinomio P de grado (Grado n1) tal que: P(xj) = yj (j=1, . . . , n). Las incógnitas del problema son los coeficientes c0, . . . , cn1 del polinomio \par

P: (x) =c0+c1x+. . .+cn1xn1=n1j=0cjxj.\par


\vspace{\baselineskip}
\textbf{Punto 7: }\par

Sea f(x) =  \( e^{x} \)  en el intervalo de [0, 1] utilice el método de lagrange y determine el tamaño del paso que me produzca un error por debajo de 10$-$ 5. ¿Es posible utilizar el polinomio de Taylor para interpolar en este caso? Verifique su respuesta.\par

\textbf{Solución: }\par

Puntos:\par



%%%%%%%%%%%%%%%%%%%% Table No: 1 starts here %%%%%%%%%%%%%%%%%%%%


\begin{table}[H]
 			\centering
\begin{tabular}{p{0.5in}p{0.5in}p{0.61in}p{0.61in}p{0.61in}p{0.61in}p{0.61in}}
\hline
%row no:1
\multicolumn{1}{|p{0.5in}}{X} & 
\multicolumn{1}{|p{0.5in}}{0} & 
\multicolumn{1}{|p{0.61in}}{0.2} & 
\multicolumn{1}{|p{0.61in}}{0.4} & 
\multicolumn{1}{|p{0.61in}}{0.6} & 
\multicolumn{1}{|p{0.61in}}{0.8} & 
\multicolumn{1}{|p{0.61in}|}{1} \\
\hhline{-------}
%row no:2
\multicolumn{1}{|p{0.5in}}{y} & 
\multicolumn{1}{|p{0.5in}}{1} & 
\multicolumn{1}{|p{0.61in}}{1.2214028} & 
\multicolumn{1}{|p{0.61in}}{1.4918247} & 
\multicolumn{1}{|p{0.61in}}{1.8221188} & 
\multicolumn{1}{|p{0.61in}}{2.2255409} & 
\multicolumn{1}{|p{0.61in}|}{2.7182818} \\
\hhline{-------}

\end{tabular}
 \end{table}


%%%%%%%%%%%%%%%%%%%% Table No: 1 ends here %%%%%%%%%%%%%%%%%%%%


\vspace{\baselineskip}


%%%%%%%%%%%%%%%%%%%% Figure/Image No: 1 starts here %%%%%%%%%%%%%%%%%%%%

\begin{figure}[H]
	\begin{Center}
		\includegraphics[width=3.93in,height=2.35in]{./media/image1.png}
	\end{Center}
\end{figure}


%%%%%%%%%%%%%%%%%%%% Figure/Image No: 1 Ends here %%%%%%%%%%%%%%%%%%%%

\par

Se buscó interpolar en x=0.5\par



%%%%%%%%%%%%%%%%%%%% Table No: 2 starts here %%%%%%%%%%%%%%%%%%%%


\begin{table}[H]
 			\centering
\begin{tabular}{p{1.33in}p{1.33in}p{1.33in}}
\hline
%row no:1
\multicolumn{1}{|p{1.33in}}{Lagrange} & 
\multicolumn{1}{|p{1.33in}}{Resultado Real} & 
\multicolumn{1}{|p{1.33in}|}{Error} \\
\hhline{---}
%row no:2
\multicolumn{1}{|p{1.33in}}{1.64872} & 
\multicolumn{1}{|p{1.33in}}{1,648721271} & 
\multicolumn{1}{|p{1.33in}|}{0,00004423 $\%$ } \\
\hhline{---}

\end{tabular}
 \end{table}


%%%%%%%%%%%%%%%%%%%% Table No: 2 ends here %%%%%%%%%%%%%%%%%%%%


\vspace{\baselineskip}

\printbibliography
\end{document}